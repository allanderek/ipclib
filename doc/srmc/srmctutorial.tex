\documentclass[10pt,a4paper]{article}
\usepackage{amssymb,amsbsy,verbatim,fancybox,ifpdf}
\usepackage{graphicx}
\usepackage[usenames]{color}

\usepackage{amssymb,amsbsy,verbatim,fancybox, ../common/pepa}
\usepackage{graphicx}
%% \usepackage{lhs2TeX}

\newcommand{\adcComment}[1]{\textbf{#1}}
\newcommand{\adcCommentl}[1]{\textbf{#1}\\}
\newcommand{\rewordthis}[1]{\textsl{#1}\\}
\newcommand{\adcTodo}{\adcComment{TODO}}

\newcommand{\ipc}{\textsf{ipc}}
\newcommand{\srmc}{\textsf{srmc}}
\newcommand{\smc}{\textsf{smc}}
\newcommand{\hydra}{\textrm{Hydra}}
\newcommand{\Condor}{\textrm{Condor}}
\newcommand{\hydrus}{\textsf{hydrus}}
\newcommand{\collate}{\textsf{collate}}
\newcommand{\puffball}{\textsf{puffball}}
\newcommand{\gnuplot}{\textsf{gnuplot}}
\newcommand{\pepa}{\textrm{PEPA}}
\newcommand{\sectref}[1]{Sect.~\ref{#1}}
\newcommand{\figref}[1]{Fig.~\ref{#1}}

% The if commands for distinguishing between the developer and user manuals
\newif\ifDeveloperVersion

\newcommand{\indexipcflag}[1]{\index{\ipc command-line flag!\texttt{#1}}}
\newcommand{\ipcflag}[1]{\indexipcflag{#1}\texttt{--\ignorespaces#1}}
\newcommand{\solver}[1]{\index{\ipc command-line flag!\texttt{--solver}!\texttt{#1}}\texttt{#1}}
\newcommand{\figSpec}[2]{
  \begin{figure}[t]
    \begin{center}
      \includegraphics[width=0.6\textwidth]{figs/#1}
      \caption[ ] {\label{fig:#1} #2}
    \end{center}
  \end{figure}
}

%%%%%%%%%%%%%%%%%%%%%%%%%%%%%%%%%%%%%%%%%%%%%%%%%%%%
%%  Url definitions
%%%%%%%%%%%%%%%%%%%%%%%%%%%%%%%%%%%%%%%%%%%%%%%%%%%%

% For now just until I work out how to check
% whether or not we are using hevea

\newif\ifusingHevea
\usingHeveafalse


\ifusingHevea
\else
\newcommand{\ahrefurl}[1]{#1}
\fi


\newcommand{\ipcstartpage}{
\ahrefurl{\url{http://homepages.inf.ed.ac.uk/s9810217/software/ipc}}}
\newcommand{\usingpuffballurl}{
\ahrefurl{\url{http://homepages.inf.ed.ac.uk/s9810217/software/puffball/usingpuffball.html}}\xspace}
\newcommand{\ipcsrctarballurl}{
\ahrefurl{\url{http://homepages.inf.ed.ac.uk/s9810217/software/ipc/ipc_src.tar.gz}}\xspace}
\newcommand{\puffballsrctarballurl}{
\ahrefurl{\url{http://homepages.inf.ed.ac.uk/s9810217/software/puffball/puffball_dist.tar.gz}}\xspace}
\newcommand{\hydrusscripturl}{
\ahrefurl{\url{http://homepages.inf.ed.ac.uk/s9810217/software/ipc/condor_example/hydrus.sh}}\xspace}
\newcommand{\collatescripturl}{
\ahrefurl{\url{http://homepages.inf.ed.ac.uk/s9810217/software/ipc/condor_example/collate.ml}}\xspace}
\newcommand{\Condorexampletarballurl}{
\ahrefurl{\url{http://homepages.inf.ed.ac.uk/s9810217/software/ipc/condor_example.tar.gz}}\xspace}
\newcommand{\airbagpepamodelurl}{
\ahrefurl{\url{http://homepages.inf.ed.ac.uk/s9810217/software/ipc/condor_example/airbag.pepa}}\xspace}
\newcommand{\HaskellWebSite}{
\ahrefurl{\url{http://www.haskell.org}}}
\newcommand{\CygwinWebSite}{
\ahrefurl{\url{http://www.cygwin.org}}}
%%%% End of Url definitions %%%%%%%%%%%


%%%%%%%%%%%%%%%%%%%%%%%%%%%%%%%%%%%%%%%%%%%%%%%%%%%%%%%%%%%%%%%%%%%%%%%%
\newcommand{\QUESTION}[1]{\textbf{\ (>> \uppercase{#1} ??)\ }}
\newcommand{\CHECKTHIS}{\textbf{\ (<<< CHECK~THIS !!!)\ }}
\newcommand{\TODO}[1]{\textbf{\ (>> \uppercase{TODO: } #1 ??)\ }}
%%%%%%%%%%%%%%%%%%%%%%%%%%%%%%%%%%%%%%%%%%%%%%%%%%%%%%%%%%%%%%%%%%%%%%%%


% A command-line sequence
\newenvironment{commandline}{\begingroup\texttt{}{}\endgroup}

%

\usepackage{srmc}


%% We must define this for the manual to operate correctly.
\newcommand{\showcommandline}[1]{
\begin{commandline}
\$ smc #1
\end{commandline}
}

\newcommand{\commandNameIpcSmc}{\smc}

%

\title{The \srmc\ Tutorial}
\author{
Allan Clark, Stephen Gilmore and Mirco Tribastone
}


\begin{document}
\maketitle


\newcommand{\esyntaxlabel}[1]{
( #1 )
}

\newcommand{\esyntaxtopline}[3]{
$ #1 $ & $ := $ & $ #2 $ & \esyntaxlabel{ #3 }
}
\newcommand{\esyntaxline}[2]{
& $ \mid $ & $ #1 $ &  \esyntaxlabel{ #2 }
}

\section*{About this document \ldots}

This document is a tutorial on the use of the
\srmc\ modelling language and associated \smc\ compiler.
This tutorial explains how to use \pepa\ for stochastic modelling
and how to use \srmc\ to describe the uncertainty in the 
underlying \pepa\ model.
No previous knowledge is assumed beyond elementary probability
theory and basic familiarity with running command-line based
applications. Experienced users of stochastic modelling tools
may wish to skim through this document if they are unfamiliar
with the \srmc\ language. For users familiar with \srmc\ there
is the separate document ``The \srmc\ Manual'' which describes
how the compiler operates and in particular the options for
controlling the behaviour of the compiler.

\section{Introduction}
When modelling a system we may be concerned with several kinds of
measurements. These measurements we understand in natural language
but they may be translated into the modelling formalism as queries.
The kinds of queries which the currently described toolchain is capable
of are:
\begin{description}
\item [Steady-state] in which we measure the long-term probability
of the system being in any of a number of given states at any given time.
\item [Transient] in which we measure the probability of being in a given
set of states a given time after the initial configuration.
\item [Passage-Time] in which we describe a sequence of observations and
compute the probability that a given time after the start of the sequence
the sequence has completed. 
\item [Rate-Measure] in which we compute the average rate at which we
may make a given observation -- such as the occurence of a particular
activity.
\end{description}

The measurements which we understand in natural language which we interpret
into the above formal queries include the following:

\begin{description}
\item[Utilisation:] A typical utilisation question is ``What
percentage of time is the server busy?''.  This is computed by summing
the relevant states of the steady-state probability distribution.

\item[Throughput:] An example here would be ``How many HTTP
requests get processed each second?''.  Here we select the relevant
states from the steady-state probability distribution and multiply by
the arrival rate.

\item[Latency:] Latency refers to the delay between the sender sending
a message and the recipient receiving it.  This is a time-dependent
property: it cannot be seen from the steady-state probability
distribution and requires transient analysis.

\item[Mean time to failure:] In reliability engineering modellers are
often concerned with the mean of the elapsed time between system
startup and system failure.  This too, is a time-dependent property
and requires transient analysis for its computation.  

\item[Service-level agreements:] An increasingly important contract in
system provisioning is the service-level agreement (SLA)\@.  A typical
SLA relates probability, time and a path through the system behaviour.
For example, the statement ``At least 95\% of requests will receive a
reply within~10 seconds'' is a typical SLA\@.

\end{description} 


\subsection{Introducing An Example}
To conclude this introduction we describe an example scenario which
one may like to model and then analyse the performance of.
This example will be expanded upon throughout this tutorial.


\begin{figure}
\begin{picture}(10,10)
% \unitlength{1cm}
\put(10.0, 120.0){\line(1,0){60.0}}    % lounge top
\put(10.0, 30.0){\line(0,1){90.0}}     % lounge left
\put(70.0, 30.0){\line(0,1){90.0}}     % lounge right
\put(40.0, 30.0){\oval(60.0, 20.0)[b]} % lounge bottom
%
\put(70.0, 120.0){\line(1,0){30.0}}    % actual bar top
\put(70.0, 80.0){\line(1,0){30.0}}     % actual bar bottom
%
\put(100.0, 120.0){\line(0,-1){70.0}}    % barroom left
\put(100.0, 120.0){\line(1,0){50.0}}     % barroom top
\put(150.0, 120.0){\line(0,-1){70.0}}    % barroom right
\put(125.0, 50.0){\oval(50.0, 20.0)[b]}  % barroom bottom
%
\put(24.0, 100.0){Lounge}                 % lounge label
\put(75.0, 100.0){Bar}                    % actual bar label
\put(105.0, 100.0){Barroom}               % barroom label
\end{picture}
\caption{
\label{figure:bar-lounge-format}
The format of the bar and lounge.
}
\end{figure}

Out scenario is that of a joint barroom and lounge.
The server is -- to begin with -- a single person who can serve
at either side, the bar or the lounge.
The lounge and bar customers occasionally order drinks.
We could measure customers individually or we could measure the barroom
customers and the lounge customers collectively.
Here is our first model, we save this to a \texttt{.srmc} file, though
at this stage it is also a simple \pepa\ model. 
That is none of the additional features offered by the \srmc\ language
are utilised.
The geography of the scenario is depicted in Figure
\ref{figure:bar-lounge-format}.

Our first model uses the approach of collectively measuring the customers
in either portion of the bar and is shown in Figure
\ref{figure:model:barroom1}.

% \begin{figure}
% \verbatiminput{models/barroom1.pepa}
% \caption{
% \label{figure:model:barroom1}
% The first model of our bar and lounge
% }
% \end{figure}


\begin{figure}
\srmcModel{
\srmcTopline{r1}
            {0.1 ;}
\\
\srmcTopline{r2}
            {0.2 ;}
\\
\srmcTopline{r3}
            {1.0 ;}
\\
\srmcTopline{Lounge}
            {(order, r1) . (serve, \srmcPassive) . Lounge ;}
\\
\srmcTopline{Barroom}
            {(order, r2) . (serve, \srmcPassive) . Barroom ;}
\\
\srmcTopline{Server}
            {(order, \srmcPassive) . (serve, r3) . Server ;}
\\
}
{
\texttt{(Lounge <> Barroom) < order, serve > Server}
}
\caption{
\label{figure:model:barroom1}
The first model of our bar and lounge
}
\end{figure}





\subsection{What's going on here}
\adcTodo
\adcComment{Explain, prefix, cooperation, passive activities and the main system equation}

\section{Analysing Our Model}
\subsection{Steady-State Analysis}
\subsection{Transient Analysis}
\subsection{Passage-Time Analysis}
\subsection{Mean-Rate Analysis}

\section{Building in Uncertainty - Rates}
\section{Building in Uncertainty - Processes}

\section{Using Namespaces}

\appendix
\section{Appendix -- Building and Installing \smc\ Compiler}
\subsection{Binary distribution}
\subsection{Source distribution}
\subsection{Darcs repository}

%%%%%%%%%%%%%%%%%%%%%%%%%%%%%%%%%%%%%%%%%%%%%%%%%%%%%%%%%%%%%%%%%%%%
%\bibliographystyle{unsrt}
%\bibliography{../common/pepa}

\end{document}

