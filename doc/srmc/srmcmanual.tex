\documentclass[10pt,a4paper]{article}
\usepackage{amssymb,amsbsy,verbatim,fancybox,ifpdf}
\usepackage{graphicx}
\usepackage[usenames]{color}


\usepackage{amssymb,amsbsy,verbatim,fancybox, ../common/pepa}
\usepackage{graphicx}
%% \usepackage{lhs2TeX}

\newcommand{\adcComment}[1]{\textbf{#1}}
\newcommand{\adcCommentl}[1]{\textbf{#1}\\}
\newcommand{\rewordthis}[1]{\textsl{#1}\\}
\newcommand{\adcTodo}{\adcComment{TODO}}

\newcommand{\ipc}{\textsf{ipc}}
\newcommand{\srmc}{\textsf{srmc}}
\newcommand{\smc}{\textsf{smc}}
\newcommand{\hydra}{\textrm{Hydra}}
\newcommand{\Condor}{\textrm{Condor}}
\newcommand{\hydrus}{\textsf{hydrus}}
\newcommand{\collate}{\textsf{collate}}
\newcommand{\puffball}{\textsf{puffball}}
\newcommand{\gnuplot}{\textsf{gnuplot}}
\newcommand{\pepa}{\textrm{PEPA}}
\newcommand{\sectref}[1]{Sect.~\ref{#1}}
\newcommand{\figref}[1]{Fig.~\ref{#1}}

% The if commands for distinguishing between the developer and user manuals
\newif\ifDeveloperVersion

\newcommand{\indexipcflag}[1]{\index{\ipc command-line flag!\texttt{#1}}}
\newcommand{\ipcflag}[1]{\indexipcflag{#1}\texttt{--\ignorespaces#1}}
\newcommand{\solver}[1]{\index{\ipc command-line flag!\texttt{--solver}!\texttt{#1}}\texttt{#1}}
\newcommand{\figSpec}[2]{
  \begin{figure}[t]
    \begin{center}
      \includegraphics[width=0.6\textwidth]{figs/#1}
      \caption[ ] {\label{fig:#1} #2}
    \end{center}
  \end{figure}
}

%%%%%%%%%%%%%%%%%%%%%%%%%%%%%%%%%%%%%%%%%%%%%%%%%%%%
%%  Url definitions
%%%%%%%%%%%%%%%%%%%%%%%%%%%%%%%%%%%%%%%%%%%%%%%%%%%%

% For now just until I work out how to check
% whether or not we are using hevea

\newif\ifusingHevea
\usingHeveafalse


\ifusingHevea
\else
\newcommand{\ahrefurl}[1]{#1}
\fi


\newcommand{\ipcstartpage}{
\ahrefurl{\url{http://homepages.inf.ed.ac.uk/s9810217/software/ipc}}}
\newcommand{\usingpuffballurl}{
\ahrefurl{\url{http://homepages.inf.ed.ac.uk/s9810217/software/puffball/usingpuffball.html}}\xspace}
\newcommand{\ipcsrctarballurl}{
\ahrefurl{\url{http://homepages.inf.ed.ac.uk/s9810217/software/ipc/ipc_src.tar.gz}}\xspace}
\newcommand{\puffballsrctarballurl}{
\ahrefurl{\url{http://homepages.inf.ed.ac.uk/s9810217/software/puffball/puffball_dist.tar.gz}}\xspace}
\newcommand{\hydrusscripturl}{
\ahrefurl{\url{http://homepages.inf.ed.ac.uk/s9810217/software/ipc/condor_example/hydrus.sh}}\xspace}
\newcommand{\collatescripturl}{
\ahrefurl{\url{http://homepages.inf.ed.ac.uk/s9810217/software/ipc/condor_example/collate.ml}}\xspace}
\newcommand{\Condorexampletarballurl}{
\ahrefurl{\url{http://homepages.inf.ed.ac.uk/s9810217/software/ipc/condor_example.tar.gz}}\xspace}
\newcommand{\airbagpepamodelurl}{
\ahrefurl{\url{http://homepages.inf.ed.ac.uk/s9810217/software/ipc/condor_example/airbag.pepa}}\xspace}
\newcommand{\HaskellWebSite}{
\ahrefurl{\url{http://www.haskell.org}}}
\newcommand{\CygwinWebSite}{
\ahrefurl{\url{http://www.cygwin.org}}}
%%%% End of Url definitions %%%%%%%%%%%


%%%%%%%%%%%%%%%%%%%%%%%%%%%%%%%%%%%%%%%%%%%%%%%%%%%%%%%%%%%%%%%%%%%%%%%%
\newcommand{\QUESTION}[1]{\textbf{\ (>> \uppercase{#1} ??)\ }}
\newcommand{\CHECKTHIS}{\textbf{\ (<<< CHECK~THIS !!!)\ }}
\newcommand{\TODO}[1]{\textbf{\ (>> \uppercase{TODO: } #1 ??)\ }}
%%%%%%%%%%%%%%%%%%%%%%%%%%%%%%%%%%%%%%%%%%%%%%%%%%%%%%%%%%%%%%%%%%%%%%%%


% A command-line sequence
\newenvironment{commandline}{\begingroup\texttt{}{}\endgroup}

%


%% We must define this for the manual to operate correctly.
\newcommand{\showcommandline}[1]{
\begin{commandline}
\$ smc #1
\end{commandline}
}

\newcommand{\commandNameIpcSmc}{\smc}

%

\title{The \srmc\ Manual}
\author{
Allan Clark, Stephen Gilmore and Mirco Tribastone
}


\begin{document}
\maketitle



\section*{About \smc}

This document is a manual for users of the \srmc\ compiler.
Users are expected to already know how to interpret their models.
A companion document, "The \srmc\ tutorial" is available.
The tutorial explains how to use \srmc\ for stochastic modelling.

\section{Input and Output}
In this section we go over the input into the \srmc\ compiler and the
output from a normal run. In Section \ref{section:command-line-options}
the options to change the behaviour of the compiler are discussed.

\adcComment{Todo: explain tool chain}

\section{Command-line Usage}
\label{section:command-line-options}
\subsection{Trivial Invocation Options}
These options only apply to the trivial
invocations of \commandNameIpcSmc.

\begin{description}
\item[-v, \ipcflag{version} ]
\indexipcflag{ version }%
The command-line:

\showcommandline{\ipcflag{version}}
will print the version of the \commandNameIpcSmc
compiler and exit.

\end{description}

\begin{description}
\item[-h, \ipcflag{help} ]
\indexipcflag{ help }%
The command-line

\showcommandline{\ipcflag{ help}}

will print an option and usage summary and exit.

\end{description}


\subsection{Output Re-directing Options}
By default the \commandNameIpcSmc\ command
will generate output in the same directory and
with a file name based on the input source name
The options in this section allow the user to
redirect the output from the
\commandNameIpcSmc\ compiler

\begin{description}
\item[\ipcflag{mod-file} FILE]
Sets the output \texttt{.mod} file which will be
the input to the run of \hydra

\end{description}

\begin{description}
\item[\ipcflag{output} FILE]
A generic output flag, this works regardless of the kind
output that \commandNameIpcSmc\ is set to produce. For example
instead of the default \hydra\ the compiler may have
been set to produce a PRISM model file.
The \ipcflag{output} can be used to set the
output file name

\end{description}

\begin{description}
\item[\ipcflag{stdout} ]
When debugging the compiler it can often be useful
for the output to be redirected to the terminal for
immediate inspection by the programmer.
The \ipcflag{stdout} sets the output file to be
the standard out. This may also prove useful for
piping the output into further processing tools.

\end{description}

\begin{description}
\item[\ipcflag{graph-output} GRAPHKIND]
When outputting a graph we wish to be able to select
what kind of output we want.
For this use the \ipcflag{graph-output} flag

\end{description}

\begin{description}
\item[\ipcflag{line-width} WIDTH]
When drawing a line graph we can select the width of a line
using the \ipcflag{line-width} flag

\end{description}


\subsection{Detecting Errors}
These options control the way that
static analysis is performed.

\begin{description}
\item[\ipcflag{staunch} ]
The flag \ipcflag{staunch} allows the compiler
to ignore warnings.
By default this flag is off and when \commandNameIpcSmc\ performs
any static-analys is over the input \pepa\ model.
The default behaviour is to treat warnings as errors so
if there are any warnings it will cause the compiler
to cancel compilation. This behaviour can be suppressed with
the \ipcflag{staunch} flag. This will cause the warnings
to still be emitted but compilation will proceed anyway.

\end{description}

\begin{description}
\item[\ipcflag{no-static-analysis} ]
The flag \ipcflag{no-static-analysis} causes
\commandNameIpcSmc\ to avoid performing the static analysis over
the \pepa\ model. It will therefore produce no
warnings or errors. Because of this compilation may fail
mysteriously and hence the user is advised only to use this
flag if they know exactly what they are doing and expect
their model to fail static-analysis for some reason but
wish to proceed to compilation anyway.

\end{description}

\begin{description}
\item[\ipcflag{allow-self-loops} ]
The flag \ipcflag{allow-self-loops} does exactly as it
is named. We allow a model containing self-loops.
This can allow the correct calculation of apparent
rates, but note that the self-looping activities
will be dropped (hence your throughput may be wrong.

\end{description}

\begin{description}
\item[\ipcflag{allow-deadlocks} ]
The flag \ipcflag{allow-deadlocks} does exactly as it
is named. We allow a model which has deadlocked states.
Mostly this is only useful for transient analysis so for
transient analysis it is the default. It is however also
possible to use this (usefully) for passage-time/end
analyses in which there is a single source state

\end{description}


\subsection{Performance Analysis Kind}
In this section the options for specifying the
different kinds of performance analysis are discussed.

\begin{description}
\item[\ipcflag{average-response} ]
Use this option to calculate the average response time
The response-time will be the time that the probe is
in the running state. This probe may be specified via
the \ipcflag{probe} argument or implicitly from a set
of source actions to a set of target actions.

\end{description}

\begin{description}
\item[\ipcflag{passage} ]
Specifies that we should perform a passage-time measurement.
This is the default. It also means that the passage-probe
is added to the model. This probe is in addition to the
master probe. It is used as the start condition of passage.
The passage-probe has two states an 'on' and 'off' state.
It will be in the 'on' state for exactly one state after the
master-probe has switched from the 'Stopped' to the 'Running'
state and 'off' otherwise.

\end{description}

\begin{description}
\item[\ipcflag{pdf} ]
When specifying a passage-time measurement state that the
probability density function should be calculated

\end{description}

\begin{description}
\item[\ipcflag{cdf} ]
When specifying a passage-time measurment state that
the cummulative distribution function of the passage
should be calculated.
This is the default however should you desire both the cdf
and the pdf, then --cdf --pdf is what you want.

\end{description}

\begin{description}
\item[\ipcflag{passage-end} ]
Specifies a special kind of passage-time calculation
in which we compare the likelihood of completing a passage
by the different stop-actions at each time.
For example we may say that if we complete a request by
time t then we have a ninety-percent chance that we
completed the request via a cache but at time 2t there is a
fifty percent chance that any request completed by then has
been served by the cache

\end{description}

\begin{description}
\item[\ipcflag{no-normalise} ]
A passage-end measurement is usually normalised against
the probability of completing the passage at all.
To suppress this and produce non-normalised results use
the \ipcflag{no-normalise} flag.
This will mean for example that you will get cdfs which
do not climb to one.

\end{description}

\begin{description}
\item[\ipcflag{steady} ]
Specifies that we should perform a steady-state analysis.
By default this will measure the probability that the probe
is in the 'Running' state.

\end{description}

\begin{description}
\item[\ipcflag{transient} ]
Specifies that we should perform a transient analysis

\end{description}

\begin{description}
\item[\ipcflag{count-measure} ACTIONS]
Specifies that we should perform a 'count' measure.
This option takes as argument the (comma separated) list of
action name we are expected to count.
This will cause hydra to return the average rate that the model
performs any of the specified actions.
Note that the specified action can be a communication label sent
by a probe specified by the user.

\end{description}

\begin{description}
\item[\ipcflag{state-measure} C-Expression]
When performing a steady-state measurement rather than
specifying the state of interest via a sequence of
actions (ie, with a probe) sometimes it is desirable to
specify it with reference to the actual states of the
components within the model. The argument to this flag
is the a state expression usually something like: 
$P3 > 0 \&\& P2 == 0$.

\end{description}

\begin{description}
\item[\ipcflag{no-measurement} ]
Suppresses the output of any measurement specification
This is generally useful if the user wishes to hand-modify
the output \texttt{.mod} file before running hydra over it.
% Therefore this option is generally run in tandem with the
% --no-run-hydra option.

\end{description}


\subsection{Probe Specification Options}
Measurement of models is performed using automatically
generated process algebra components called
\emph{stochastic probes}
This section describes the options which control
the performance measure specification probes which
are added to the model.
% For more information on this please see section ...

\begin{description}
\item[-p, \ipcflag{probe} PROBESPEC]
The flag \ipcflag{ probe } with the shortened version
\texttt{p} is used to give a full probe specification.
% In the probe specification language given in section ??

\end{description}

\begin{description}
\item[\ipcflag{no-master} ]
The flag \ipcflag{ no-master } is used to specify that
no master probe % as described in section ...
should be automatically added to the model.

\end{description}

\begin{description}
\item[-s, \ipcflag{source} ACTIONS]
The flags \ipcflag{ source } and \ipcflag{ target }
with the short versions \texttt{s} and \texttt{t}
respectively, set the state switching actions used
in the master probe. If other probes are added using the
\ipcflag{ probe } option then they may perform immediate
communication actions which are specified in the source and
target action list.
Both the \ipcflag{ source } and \ipcflag{ target } flags
accept as argument a comma separated list of action names.

\end{description}

\begin{description}
\item[-t, \ipcflag{target} ACTIONS]
see \texttt{--source}
\end{description}

\begin{description}
\item[\ipcflag{source-cond} CONDITION]
The flags \ipcflag{source-cond} and \ipcflag{target-cond}
allow the specification of a passage-time measurement 
with respect to state conditions rather than action
observations.

\end{description}

\begin{description}
\item[\ipcflag{target-cond} CONDITION]
see \texttt{--source-cond}
\end{description}


\subsection{Aggregation}
The options in this section control the aggregation
of components.
Currently in \commandNameIpcSmc\ a component may
only be aggregated if the user explicitly writes it
as a component array of the form $P[N]$.

\begin{description}
\item[\ipcflag{aggregate} ]
Tells \commandNameIpcSmc\ to aggregate process arrays

\end{description}

\begin{description}
\item[\ipcflag{no-aggregate} ]
Tells \commandNameIpcSmc\ \textbf{not} to aggregate
process arrays. Therefore $P[3][a]$ will be translated
into the form $P <a> P <a> P$.

\end{description}

\begin{description}
\item[\ipcflag{limit} SIZE]
Tells the state space generator to quit after reaching a
given limit in the state space size

\end{description}


\subsection{Running Hydra}
This section details options used for running
the hydra tool after processing of the model by \ipc.

\begin{description}
\item[\ipcflag{run-hydra} ]
The flag \ipcflag{run-hydra} causes \commandNameIpcSmc\ to
automatically run the hydra tool on the produced
\texttt{.mod} file.
This option has been deprecated please use the
\ipcflag{hydra-stage} option described below.

\end{description}

\begin{description}
\item[\ipcflag{hydra} PATH]
If the hydra tool is not installed in a standard
location the path to the \hydra\ executable can
be given as an argument to the \ipcflag{hydra} option.

\end{description}

\begin{description}
\item[\ipcflag{hydra-stage} STAGE]
Choose at which stage we should stop using ipc
and switch over to hydra.
 This option replaces the older \ipcflag{run-hydra} option.
There are currently two stages which may be specified:
\begin{itemize}
\item 'mod' this outputs a hydra model file in the original
format. The full state space is not computed by \ipc.
\item 'flat-mod' here the hydra model file contains a full
state space and as such does not suffer from a problem of
very large rate expressions.
\end{itemize}

\end{description}


\subsection{Prism Model Files}
At some point we wish to be able to generate prism
model files from Pepa model descriptions.
All of the options described in this section should be
considered experimental and not for general use.

\begin{description}
\item[\ipcflag{prism} STAGE]
\ipcflag{prism} this is an experimental option to produce
a prism model.
When using this one specifies a stage to which we with to
to compile the pepa model to.
Currently there are two supported options "trans" and
"model". The 'model' option should be considered not to be
working since it is in a very early stage of development
The 'trans' stage will compile the model down to an
explicit state space which prism can then import.

\end{description}

\begin{description}
\item[\ipcflag{prism-file} FILE]
The \ipcflag{prism-file} option only has an effect if the
\texttt{--prism} flag is set. It redirects the output
prism model to the specified file.

\end{description}

\begin{description}
\item[\ipcflag{prism-options} OPTIONS]
The \ipcflag{prism-options} flag has the effect to
append the provided string on to the end of the
the prism command

\end{description}


\subsection{Dizzy Model Files}
At some point we wish to be able to generate dizzy
model files from Pepa model descriptions.
The options in this section describe the control of
the \pepa\ $\rightarrow$ dizzy translation.

\begin{description}
\item[\ipcflag{dizzy} ]
\ipcflag{dizzy} this is an experimental option to produce
a dizzy model. This should not be considered working

\end{description}

\begin{description}
\item[\ipcflag{dizzy-file} FILE]
The \ipcflag{dizzy-file} option only has an effect if the
\texttt{--dizzy} flag is set. It redirects the output
dizzy model to the specified file.

\end{description}


\subsection{Extra Output}
The model undergoes various transformations and
augmentations on its way to being compiled.
Sometimes it is helpful to see these intermediate
stages. The options in this section allow the user
to specify that \commandNameIpcSmc\ should
include a particular intermediate model in the output,
generally inside comments of the output file.

\begin{description}
\item[\ipcflag{show-simplified} ]
Specifies that \commandNameIpcSmc\ should show the
simplified model. This is a model without any of the
syntactic sugar which the user may use for convenience
but are unnecessary for the compilation procedure

\end{description}

\begin{description}
\item[\ipcflag{show-probed} ]
Specifies that the model with the measurement probe
components added and then simplified is shown.

\end{description}


\subsection{Logging Options}
The options in this section affect what, if anything
is logged and how it is logged

\begin{description}
\item[\ipcflag{show-log} ]
Specifies that the logging information should be shown to stdout

\end{description}

\begin{description}
\item[\ipcflag{log} categories]
The flag \ipcflag{log} specifies that a log should be kept
Each log entry is associated with a name and the argument
to the \ipcflag{log} flag is a name specifying that that
entry should be output to the log file.
The name 'all' specifies that all log entries
should be recorded in the log-file.
Additionally the 'user' name outputs to the log items
deemed relevant to the user while 'developer' is a set
of log items relevant for the developer.

\end{description}


\subsection{Miscellaneous Options}
This section describes options which do not fit
under any of the previous sub-sections.

\begin{description}
\item[\ipcflag{steady-mean} ]
No manual entry but the usage information states:
use the 'mean' estimator in a steady-state measure
\end{description}

\begin{description}
\item[\ipcflag{steady-variance} ]
No manual entry but the usage information states:
use the 'variance' estimator in a steady-state measure
\end{description}

\begin{description}
\item[\ipcflag{steady-stddev} ]
No manual entry but the usage information states:
use the 'stddev' estimator in a steady-state measure
\end{description}

\begin{description}
\item[\ipcflag{steady-distrib} ]
No manual entry but the usage information states:
use the 'distribution' estimator in a steady-state measure
\end{description}

\begin{description}
\item[\ipcflag{start-time} TIME]
No manual entry but the usage information states:
specify a time at which to start a performance measure eg passage-time
\end{description}

\begin{description}
\item[\ipcflag{stop-time} TIME]
No manual entry but the usage information states:
specify a time at which to stop a performance measure eg passage-time
\end{description}

\begin{description}
\item[\ipcflag{time-step} TIME]
No manual entry but the usage information states:
specify a the time steps for a performance measurement
\end{description}

\begin{description}
\item[\ipcflag{solver} SOLVER]
No manual entry but the usage information states:
specify which solution method to use/specify to hydra
\end{description}

\begin{description}
\item[\ipcflag{rate} DOUBLE]
No manual entry but the usage information states:
Override/specify a rate value on the command-line
\end{description}

\begin{description}
\item[\ipcflag{rename-proc} P=s]
No manual entry but the usage information states:
cause a renaming on the given process within the model
\end{description}

\begin{description}
\item[\ipcflag{rename-rate} r=s]
No manual entry but the usage information states:
cause a renaming on the given rate within the model
\end{description}

\begin{description}
\item[\ipcflag{transform-rule} RULE]
Provide a transformation rule with which
to automatically transform the PEPA model

\end{description}

\begin{description}
\item[\ipcflag{prioritise} ACTIONS]
No manual entry but the usage information states:
increase the priority of the given actions
\end{description}

\begin{description}
\item[\ipcflag{dot-file} ]
Produce a .dot file of the state space and run the dot program over it
\end{description}

\begin{description}
\item[\ipcflag{no-reduce-vanishing} ]
When producing a state space for external output
do not remove the vanishing states

\end{description}

\begin{description}
\item[\ipcflag{no-reduce-rate-exps} ]
No manual entry but the usage information states:
do not reduce the rate expressions
\end{description}

\begin{description}
\item[\ipcflag{hide-non-coop} ]
No manual entry but the usage information states:
hide any activities which a component performs but does not cooperate over
\end{description}

\begin{description}
\item[\ipcflag{process-num} NUM]
No manual entry but the usage information states:
provide a process number which is used to select rates and processes
\end{description}

\begin{description}
\item[\ipcflag{fsp} ]
The \ipcflag{fsp} flag is an experimental option to produce
an LTSA model. This should not be considered working

\end{description}

\begin{description}
\item[\ipcflag{states-size} ]
The flag \ipcflag{states-size} informs \commandNameIpcSmc\ to
print out the size of the state space of the given model

\end{description}

\begin{description}
\item[\ipcflag{estimate-size} ]
The flag \ipcflag{estimate-size} asks \commandNameIpcSmc\ to simply
estimate the final state space size of the input model

\end{description}

\begin{description}
\item[\ipcflag{compare-pepato} ]
The flag \ipcflag{compare-pepato} informs \commandNameIpcSmc\ to
produce a steady-state analysis and run pepato over
the model to also produce a steady-state analysis and
compare the two reports.

\end{description}

\begin{description}
\item[\ipcflag{experimental} ]
The flag \ipcflag{expermental} is for developers only
This is a generic flag meaning:
"enable a new approach/version of .."
Generally only to be used by ipc developers.
So for example a new approach to state space generation,
rather than inventing a new flag
"--use-new-state-space-gen" we can just test for this.
If later we decide that we do actually wish to have both
*then* we can invent two flags for it.

\end{description}





\section{Concrete Syntax}
The concrete syntax for srmc model files is given in Figure
\ref{figure:syntax:srmcfiles}.
The parts of the grammar which are written down within an srmc file
are highlighted in \textcolor{blue}{blue}, everything else is a 
bnf style operator.

%%%%%%%%%%%%%%%%%%%%%%%%%%%%%%%%%%%%%%%%%%%%%%%%%%%%%%%%%%
%% General commands for writing a grammar
\newcommand{\esyntaxlabel}[1]{
( #1 )
}

\newcommand{\esyntaxtopline}[3]{
$ #1 $ & $ := $ & $ #2 $ & \esyntaxlabel{ #3 }
}
\newcommand{\esyntaxline}[2]{
& $ \mid $ & $ #1 $ &  \esyntaxlabel{ #2 }
}


%%%%%%%%%%%%%%%%%%%%%%%%%%%%%
% The commands for the grammar
\newcommand{\model}{ model }
\newcommand{\definitionList}{def\_list}
\newcommand{\definition}{def}

\newcommand{\ratedef}{rate\_def}
\newcommand{\processdef}{process\_def}
\newcommand{\namespace}{name\_space}

% Rates
\newcommand{\rateId}{rate\_ident}
\newcommand{\qualifiedRateId}{Q\rateId}
\newcommand{\rateSet}{rate\_set}
\newcommand{\rate}{rate}
\newcommand{\rateExp}{rate\_exp}
\newcommand{\commaRateExps}{rate\_list}
\newcommand{\rateOp}{rOper}
\newcommand{\rateCond}{rCond}


% Processes
\newcommand{\processId}{process\_ident}
\newcommand{\qualifiedProcessId }{ Q\processId }
\newcommand{\processSet}{process\_set}
\newcommand{\process}{process}
\newcommand{\commaProcesses}{process\_list}
\newcommand{\prefix}{prefix}
\newcommand{\condOper}{condOper}

% Conditional behaviours
\newcommand{\beCondition}{condition}

% Name spaces
\newcommand{\namespaceId}{space\_ident}
\newcommand{\qualifiedNameSpaceId}{Q\namespaceId}

% Actions
\newcommand{\actionName}{action\_name}
\newcommand{\actionList}{action\_list}

% Numbers
\newcommand{\forgivFloat}{int\_or\_float}

%Identifiers
\newcommand{\lowerid}{lower\_ident}
\newcommand{\upperid}{upper\_ident}
\newcommand{\qualifier}{qualifier}
\newcommand{\qualifiedLower}{Q\lowerid}
\newcommand{\qualifiedUpper}{Q\upperid}

\newcommand{\lowers}{\concrete{a}-\concrete{z}}
\newcommand{\uppers}{\concrete{A}-\concrete{Z}}
\newcommand{\digits}{\concrete{0}-\concrete{9}}

\newcommand{\systemEquation}{system}

% bnf operators
\newcommand{\optional}[1]{\langle #1 \rangle}
\newcommand{\oneOrMore}[1]{ #1^+ }
\newcommand{\bnfBracket}[1]{ \{ #1 \} }
\newcommand{\oneOrMoreBracketed}[1]{\oneOrMore{\bnfBracket{#1}}}
\newcommand{\zeroOrMore}[1]{ #1^* }
\newcommand{\zeroOrMoreBracketed}[1]{\zeroOrMore{\bnfBracket{#1}}}


%Parts of the actual concrete grammar
\newcommand{\concrete}[1]{\textcolor{blue}{\texttt{#1}}}
\newcommand{\concreteBraces}[1]{\concrete{ \{ } #1 \concrete{ \} }}
\newcommand{\equals}{\concrete{=}}
\newcommand{\comma}{\concrete{,}}
%%%%%%%%%%%%%%%%%%%%%%%%%%%%%
% The grammar figure
\begin{figure}[htb]
\begin{tabular}{lclr}
% Models
\esyntaxtopline{ \model }
               { \definitionList\ \systemEquation }
               { model }
\\
% Definition lists
\esyntaxtopline{ \definitionList }
               { \oneOrMoreBracketed{ \definition \concrete{;} } }
               { definition list }
\\
%% \esyntaxline   { \definition }
%%                { one definition }
%% \\
% Definitions
\esyntaxtopline{ \definition }
               { \ratedef }
               { rate definition }
\\
\esyntaxline   { \processdef }
               { process definition }
\\
\esyntaxline   { \namespace }
               { name space }
\\

% RateDefinitions
\esyntaxtopline{ \ratedef }
               { \rateId\ \equals\ \rateSet }
               { rate definition }
\\
\esyntaxtopline{ \rateSet }
               { \rateExp }
               { Single rate }
\\
\esyntaxline   { \concreteBraces{ \commaRateExps } }
               { set of rates }
\\
\esyntaxtopline{ \commaRateExps }
               { \rateExp\ \oneOrMoreBracketed{ \comma\ \rateExp } }
               { set of rates }
\\
% Process Definitions
\esyntaxtopline{ \processdef }
               { \optional{\concrete{ \# } } \processId
                 \concrete{ = } \process }
               { process definition }
\\
% Name space selection
\esyntaxtopline{ \namespace }
               { \namespaceId :: \concrete{ = }
                                 \concreteBraces{ \qualifiedNameSpaceId 
                                                  \oneOrMoreBracketed{ \comma\ \qualifiedNameSpaceId }
                                                }
               }
               { Name space selection}
\\
% Name space
\esyntaxtopline{ \namespace }
               { \namespaceId :: \concreteBraces{ \definitionList } }
               { Name space }
\\
% Components
\esyntaxtopline{ \process }
               { \qualifiedProcessId }
               { named process }
\\
\esyntaxline   { \prefix
                 \ \process
               }
               { Prefix components }
\\
\esyntaxline   { \concrete{ if }
                 \ \beCondition
                 \ \concrete{ then }
                 \ \prefix
                 \ \process
               }
               { conditional behaviour }
\\
\esyntaxline   { \process
                 \concrete{ + }
                 \process
               }
               { Choice }
\\
\esyntaxline   { \process
                 \concrete{ < }
                 \optional{ \actionList }
                 \concrete{ > }
                 \process
               }
               { Cooperation }
\\
\esyntaxline   { \process \concrete{ / }
                          \concreteBraces{ \actionList }
               }
               { Hiding }
\\
\esyntaxline   { \qualifiedProcessId 
                 \concrete{ [ }
                 \concrete{ \digits }
                 \concrete{ ] }
                 \optional{ \concrete{ [ }
                            \actionList
                            \concrete{ ] }
                          }
               }
               { process array }
\\
% Prefixes
\esyntaxtopline{ \prefix }
               { \concrete{ ( } \actionName
                 \concrete{,}
                 \rate
                 \concrete{ ) . }
               }
               { timed prefix }
\\
\esyntaxline   { \actionName 
                 \concrete{ . }
               }
               { immediate prefix }
\\

% conditions
\esyntaxtopline{ \beCondition }
               { \qualifiedRateId
                 \ \condOper
                 \ \rateExp
               }
               { behaviour condition }
\\

% condition operators
\esyntaxtopline{ \condOper }
               {      \concrete{ < }
                 \mid \concrete{ > }
                 \mid \concrete{ = }
                 \mid \concrete{ >= }
                 \mid \concrete{ <= }
                 \mid \concrete{ <> }
               }
               { condition operators }
\\
% Rates
\esyntaxtopline{ \rate }
               { \concrete{ infty } \mid \concrete{ \_ } }
               { passive }
\\
\esyntaxline   { \rateExp }
               { timed rate expression }
\\
\esyntaxline   { \forgivFloat
                 \concrete{ : immediate } 
               }
               { immediate }
\\
% Rate Expressions
\esyntaxtopline{ \rateExp }
               { \qualifiedRateId }
               { named rate }
\\
\esyntaxline   { \forgivFloat }
               { literal float }
\\
\esyntaxline   { \rateExp\ \rateOp\ \rateExp }
               { binary op expression }
\\
\esyntaxline   { \qualifiedProcessId }
               { process rate }
\\
\esyntaxline   { \concrete{ if }
                 \rateCond
                 \concrete{ then }
                 \rateExp 
                 \concrete{ else }
                 \rateExp
               }
               { conditional rate }
\\
               
% Rate Expression binary operators
\esyntaxtopline{ \rateOp }
               { \concrete{ + }
                 \mid
                 \concrete{ - }
                 \mid
                 \concrete{ / }
                 \mid
                 \concrete{ * }
               }
               { rate operators }
\\
% Rate conditions
\esyntaxtopline{ \rateCond }
               { \qualifiedProcessId }
               { Process present }
\\

% Forgiving Floats
\esyntaxtopline{ \forgivFloat }
               { \concrete{ \digits }
                 \optional{ \concrete{. \digits } }
               }
               { integer or float }
\\
% Action List
\esyntaxtopline{ \actionList }
               { \actionName \oneOrMoreBracketed{ \concrete{ , }
                                                  \actionName
                                                }
               }
               { comma action list }
\\

% Identifiers
\esyntaxtopline{ \actionName }
               { [\lowers]\zeroOrMore{[\lowers\ \uppers\ \digits]} }
               { lower identifier }
\\
\esyntaxtopline{ \rateId }
               { [\lowers]\zeroOrMore{[\lowers\ \uppers\ \digits]} }
               { lower identifier }
\\
\esyntaxtopline{ \processId }
               { [\uppers]\zeroOrMore{[\lowers\ \uppers\ \digits]} }
               { upper identifier }
\\
\esyntaxtopline{ \namespaceId }
               { [\uppers]\zeroOrMore{[\lowers\ \uppers\ \digits ]} }
               { all upper }
\\
\esyntaxtopline{ \qualifier }
               { \zeroOrMoreBracketed{ \namespaceId \concrete{::} } }
               { name qualifier }
\\
\esyntaxtopline{ \qualifiedRateId }
               { \qualifier\ \rateId }
               { qualified rate name }
\\
\esyntaxtopline{ \qualifiedProcessId }
               { \qualifier\ \processId }
               { qualified process name }
\\
\esyntaxtopline{ \qualifiedNameSpaceId }
               { \qualifier\ \namespaceId }
               { qualified process name }
\\
% Finallly the system description
\esyntaxtopline{ \systemEquation }
               { \process }
               { main system }
\\
\end{tabular}
\caption{
\label{figure:syntax:srmcfiles}
The concrete syntax for srmc model files
}
\end{figure}


\section{Notes, Comments and Questions}
\begin{itemize}
\item
There are three kinds of identifiers.
An upper-ident these begin with an upper-case letter and are used for
process names, eg $Client$.
A lower-ident these begin with a lower-case letter and are used for rate names,
eg $lambda$.
An all-upper-ident, these begin with and only contain upper-case letters though 
they may also have digits in them, eg $EDINBURGH$.
A qualified name, can end in any of the three kinds of identifiers, but it must
consist of a (possibly empty) chain of all-upper-idents followed
by double-colons, eg
$EDINBURGH::INFORMATICS::LFCS::Client$ is a qualified process name.
$EDINBURGH::INFORMATICS::LFCS::lambda$ is a qualified rate name.

\item
One may wish to combine the $conditional$ syntax for conditional behaviour and
functional rates. However the two te⟨st for distinct things. The first tests are
generally on rates and may be compiled out before the model is solved.
Functional rates on the other hand generally consist of process states and 
are a dynamic test which cannot be compiled out
(technically, cannot be evaluated at compile-time).

% \item
% Do we still wish to retain the legacy (optional) \# syntax
% to begin a process definition?
% The verdict seems to be yes.
\end{itemize}

%%%%%%%%%%%%%%%%%%%%%%%%%%%%%%%%%%%%%%%%%%%%%%%%%%%%%%%%%%%%%%%%%%%%
%\bibliographystyle{unsrt}
%\bibliography{../common/pepa}

\end{document}

